%%%%%%%%%%%%%%%%%%%%%%%%%%%%%%%%%%%%%%%%%
% "ModernCV" CV and Cover Letter
% LaTeX Template
% Version 1.11 (19/6/14)
%
% This template has been downloaded from:
% http://www.LaTeXTemplates.com
%
% Original author:
% Xavier Danaux (xdanaux@gmail.com)
%
% License:
% CC BY-NC-SA 3.0 (http://creativecommons.org/licenses/by-nc-sa/3.0/)
%
% Important note:
% This template requires the moderncv.cls and .sty files to be in the same 
% directory as this .tex file. These files provide the resume style and themes 
% used for structuring the document.
%
%%%%%%%%%%%%%%%%%%%%%%%%%%%%%%%%%%%%%%%%%

%----------------------------------------------------------------------------------------
%	PACKAGES AND OTHER DOCUMENT CONFIGURATIONS
%----------------------------------------------------------------------------------------

\documentclass[11pt,a4paper,sans]{moderncv} % Font sizes: 10, 11, or 12; paper sizes: a4paper, letterpaper, a5paper, legalpaper, executivepaper or landscape; font families: sans or roman

\moderncvstyle{casual} % CV theme - options include: 'casual' (default), 'classic', 'oldstyle' and 'banking'
\moderncvcolor{blue} % CV color - options include: 'blue' (default), 'orange', 'green', 'red', 'purple', 'grey' and 'black'

\usepackage{lipsum} % Used for inserting dummy 'Lorem ipsum' text into the template

\usepackage{multicol} % Used for itemizing items in multiple columns

\usepackage[scale=0.75]{geometry} % Reduce document margins
%\setlength{\hintscolumnwidth}{3cm} % Uncomment to change the width of the dates column
%\setlength{\makecvtitlenamewidth}{10cm} % For the 'classic' style, uncomment to adjust the width of the space allocated to your name
\usepackage[T1]{fontenc}
\usepackage{lmodern}
%----------------------------------------------------------------------------------------
%	NAME AND CONTACT INFORMATION SECTION
%----------------------------------------------------------------------------------------

\firstname{Joan} % Your first name
\familyname{Cano Aladid} % Your last name

% All information in this block is optional, comment out any lines you don't need
\title{Curriculum Vitae}
\address{C/ Luis Serra nº2, Esc.2, 4º, 4ª}
\mobile{(+34) 686500085}
\email{joan@liberam.es}

%----------------------------------------------------------------------------------------

\begin{document}

\makecvtitle% Print the CV title

%----------------------------------------------------------------------------------------
%	EDUCATION SECTION
%----------------------------------------------------------------------------------------

\cventry{2016 - 2017}{Máster en TIG}{Sistemas de Información Geográfica y Teledetección}{Zaragoza}{}{}
\cventry{2012 - 2016}{Grado en Geografía y Ordenación del Territorio}{}{Universidad de Alicante}{}{}

%----------------------------------------------------------------------------------------
%   EXPERIENCE SECTION
%----------------------------------------------------------------------------------------

\section{Experiencia laboral}
\cventry{03/2022 - Actualidad}{Gestor}{LIBERAM TECHNOLOGIES, S.L.}{Jaca}{Huesca}{Desarrollo de trabajos relacionados con la geomática}
\cventry{07/2017 - 01/2022}{Técnico en geomática}{3DSCANNER}{Zaragoza}{}{Desarrollo de trabajos de topografía, escaneado y fotogrametría en el ámbito del Patrimonio, Medio Ambiente, Ingeniería e Industria}
\cventry{07/2017 - 01/2022}{Soporte técnico}{Tecnitop}{Zaragoza}{}{Soporte técnico de Leica en las soluciones HDS. Soporte técnico oficial de drones SenseFly, DJI, Parrot y Flyability}
\cventry{06/2019}{Instructor}{Fundación laboral de la construcción}{Zaragoza}{}{Fotogrametría aérea y terrestre para construcción y obra civil}
\cventry{02/2016 - 04/2016}{Prácticas de empresa en el Ayuntamiento de Sella y Laboratorio de Geomática}{Universidad de Alicante}{}{}{Diseño de bases de datos para la gestión del cementerio municipal. Infracciones urbanísticas. Cartografía municipal}

%----------------------------------------------------------------------------------------
%   HONOURS AND AWARDS SECTION
%----------------------------------------------------------------------------------------

\section{Premios y Becas}
\cventry{2018}{Mención de excelencia en el Trabajo de Fin de Máster}{Calibración de sensores multiespectrales y su aplicación en UAV. Caso de estudio: Análisis de la cámara Sequoia}{}{}{}

%----------------------------------------------------------------------------------------
%	COURSEWORK SECTION
%----------------------------------------------------------------------------------------

\section{Habilidades}
\vspace{-5mm}
\cventry{}{}{}{}{}{
\begin{multicols}{2}
\begin{itemize}
	\item Sistemas de Información Geográfica
	\item CAD
    \item Teledetección
    \item Piloto de drones
    \item Geoservicios web y móvil
    \item Fotogrametría aérea y terrestre
    \item Láser escáner 3D
    \item Edición audiovisual
    \item Modelado 3D
    \item Cartografía
    \item Gestión y ejecución de proyectos
\end{itemize}
\end{multicols}
}

%----------------------------------------------------------------------------------------
%   LANGUAGE SECTION
%----------------------------------------------------------------------------------------

\section{Idiomas}

\cvitem{Inglés}{\textbf{nivel B1-PET}}
\cvitem{Castellano}{\textbf{alto}}
\cvitem{Valenciano}{\textbf{alto (lengua materna)}}

%\newpage
%----------------------------------------------------------------------------------------

\section{Cursos y seminarios}

\cventry{2024}{UAVs: toma de datos con cámaras multiespectrales y exploración 3D del territorio}{Máster TIG Unizar}{2h}{Impartido}{}
\cventry{2024}{Técnicas de documentación geométrica para la generación de MDT}{Máster riesgos naturales Universidad de Alicante}{8h}{Impartido}{}{}
\cventry{2024}{Iniciación al uso de drones: Legislación vigente, tipos de UAS y aplicaciones}{Colegio Geógrafos Valencia}{2h}{Impartido}{}{}
\cventry{2023}{Especialista en nivología y avalanchas II: Gestor de datos}{INAEM}{100h}{Recibido}{}{}
\cventry{2023}{Topografía digital y proyectos de obra civil con MDT}{INAEM}{250h}{Recibido}{}{}
\cventry{2023}{International Workshop Geomatics Methodologies in Archaeology and Cultural Heritage Research}{UIMP}{Asistente}{}{}{}
\cventry{2023}{UAVs: toma de datos con cámaras multiespectrales y exploración 3D del territorio}{Máster TIG Unizar}{2h}{Impartido}{}{}
\cventry{2023}{Técnicas de documentación geométrica para la generación de MDT}{Máster riesgos naturales Universidad de Alicante}{8h}{Impartido}{}
\cventry{2023}{Calificación como radiofonista para pilotos remotos}{10h}{Recibido}{}{}
\cventry{2023}{Curso STS-ES-01/STS-ES-02}{}{Recibido}{}{}
\cventry{2022}{Habilitación categoría abierta UAV}{}{Recibido}{}{}
\cventry{2022}{Reparación de drones Elios (Flyability)}{Laussane, Suiza}{24h}{Recibido}{}
\cventry{2022}{Técnicas de documentación geométrica 3D avanzada para el R3AD}{8h}{Impartido}{}{}
\cventry{2020}{Curso Davinci Resolve}{18h}{Recibido}{}{}
\cventry{2019}{Formación Leica escáner láser 3D}{20h}{Recibido}{}{}
\cventry{2019}{Curso de instructor de RPAS, por EPRYD}{Escuela ULM número de registro 2018004169. 15 horas}{Recibido}{}{}
\cventry{2019}{Comunicación oral en ISMAR10: Monitorización de los recursos hidrológicos nivales}{el glaciar de Monte Perdido (Huesca)}{Impartido}{}{}
\cventry{2019}{Curso de Blender}{Postprocesado fotogramétrico y reconstrucción virtual. 60h}{Recibido}{}{}
\cventry{2018}{Jornada sobre el uso de drones y escáner 3D aplicada a PRL para Asepeyo}{Impartido}{}{}{}
\cventry{2018}{Jornada sobre el uso de drones y escáner 3D para La Junta de Castilla y León}{Impartido}{}{}{}
\cventry{2017}{SenseFly Trainer Certificate en eBeeX}{Recibido}{}{}{}
\cventry{2017}{Curso de manejo de Pix4D}{10 horas}{Recibido}{}{}
\cventry{2017}{Curso Teórico y Práctico de aeronave pilotada por control remoto de clase avión}{eBee}{Recibido}{}{}
\cventry{2017}{Curso avanzado de Piloto de Drones por el Real Aeroclub de Zaragoza}{60 horas}{Recibido}{}{}
\cventry{2017}{Microviñas}{una herramienta para el ecosistema empresarial rural}{Recibido}{}{}
\cventry{2015}{Bases de datos espaciales}{PostGis}{Recibido}{}{}


\section{Voluntariado}

\cventry{2017 - Actualidad}{Mapeado Colaborativo}{grupo residente en Zaragoza Activa}{Zaragoza}{}{}{}
\cventry{2018}{Miembro del grupo DroneMapZ}{proyecto ciudadano en las Convocatorias Cesar de Etopia}{Zaragoza}{}{}
\cventry{2016}{Voluntariado en Cobán (Guatemala)}{Proyecto de cartografía colaborativa basado en OpenStreetMap}{Universidad de Alicante}{}{}

%\newpage

\section{Principales trabajos realizados}


\cventry{2024}{Generación de gemelos digitales de varias localizaciones para próxima serie de Netflix}{Ámbito nacional}{}{}{}

\cventry{2024}{Gira 2024 con el colectivo de danza contemporánea Qabalum}{Ámbito internacional}{}{}{}

\cventry{2024}{Realizando el levantamiento cartográfico del tramo A-92 con G-30 para nuevo enlace}{}{Albolote}{}{}

\cventry{2024}{Realizando trabajos de documentación geométrica, espacial y fotogramétrica de varios fortines de la Guerra Civil}{Brunete}{}{}{}

\cventry{2023}{Realizando levantamientos topográficos con UAV para tendido aéreo y subterráneo}{Muel}{Zaragoza}{}{}

\cventry{2023}{Realizando levantamientos topográficos con UAV del tramo Bobadilla para la redacción del proyecto de construcción de adecuación de gálibo de las estructuras de la autopista ferroviaria Algeciras - Zaragoza}{Córdoba}{}{}{}

\cventry{2022}{Realizando trabajos de escáner láser 3D, fotogrametría aérea y terrestre}{Documentación geométrica y fotogramétrica de excavación arqueológica}{Bronchales}{}{}

\cventry{2022}{Realizando trabajos de digitalización para restauración}{Documentación geométrica y fotogramétrica del Oratorio de la Villa}{Madrid}{}{}

\cventry{2022}{Impartiendo cursos de aplicaciones UAS: Fotogrametría y teledetección}{European Flyers, Madrid}{}{}{}

\cventry{2021}{Realizando trabajos de enlace con la geodesia espacial, escaneado láser 3D y fotogrametría terrestre}{Documentación geométrica y fotogramétrica de La Grotte du Gargas}{Francia}{}{}

\cventry{2021}{Realizando trabajos de enlace con la geodesia espacial, escaneado láser 3D y fotogrametría aérea}{Documentación geométrica y fotogramétrica de la presa de Santolea}{Teruel}{}{}

\cventry{2021}{Reconstrucción tridimensional y fotorealista de antiguos hornos de cerámica para documentación arqueológica}{Fotogrametría y escaneado láser}{Samarkanda}{Uzbekistán}{}

\cventry{2021}{Realizando trabajos de digitalización 3D y toma de datos multi e hiperespectral de un pavés}{Documentación geométrica, fotogramétrica y espectral}{Tortosa}{}{}

\cventry{2021}{Realizando trabajos de digitalización 3D de varias estatuas}{Documentación geométrica y fotogramétrica de estatuas}{Madrid}{}{}

\cventry{2021}{Realizando trabajos de enlace con la geodesia espacial, escaneado láser 3D y fotogrametría terrestre}{Documentación geométrica y fotogramétrica de la cueva de la Fuente del Trucho}{Huesca}{}{}

\cventry{2020}{Realizando trabajos de programación web}{Desarrollo en html5 para la visualización y estudio de grabados rupestres}{La Grotte du Roucadour}{}{}

\cventry{2020}{Realizando trabajos de enlace con la geodesia espacial, escaneado láser 3D y fotogrametría terrestre}{Documentación geométrica y fotogramétrica de la Cueva de Maltravieso}{Cáceres}{}{}

\cventry{2020}{Realizando trabajos de enlace con la geodesia espacial, escaneado láser 3D}{Documentación geométrica de la cueva Judge's Cave}{}{Gibraltar}{}

\cventry{2020}{Realizando trabajos de escaneado láser 3D y fotogrametría}{Documentación geométrica y fotogramétrica de la casa de Goya}{Quinta del sordo}{Madrid}{}

\cventry{2020}{Realizando trabajos de enlace con la geodesia espacial, escaneado láser 3D y fotogrametría}{Documentación geométrica y fotogramétrica de un hallazgo arqueológico en un abrigo}{Barranco Gomez}{}{}

\cventry{2020}{Realizando trabajos de enlace con la geodesia espacial, escaneado láser 3D y fotogrametría}{Documentación geométrica y fotogramétrica de un torreón en Alcañiz}{}{}{}

\cventry{2019}{Levantamiento topográfico y delineación 3D}{Municipios de Artiès, Baquèira, Salardú y tresdós}{Vall d’Aran}{}{}

\cventry{2019}{Realizando trabajos de enlace con la geodesia espacial, escaneado láser 3D y drones}{Documentación geométrica y fotogramétrica del Dolmen de Guadalperal y Agustóbriga}{}{}{}

\cventry{2019}{Realizando trabajos de enlace con la geodesia espacial, escaneado láser 3D, fotogrametría aérea y terrestre}{Documentación geométrica, espacial y multiespectral del Castillo de Santed}{}{}{}

\cventry{2019}{Realizando trabajos de enlace con la geodesia espacial, escaneado láser 3D y fotogrametría terrestre}{Documentación geométrica y espacial de la Presa Romana de Muel}{}{}{}

\cventry{2019}{Vuelo fotogramétrico con dron}{Levantamiento topográfico para subestación eléctrica y nuevas vías de transporte}{Moneva y Moyuela}{}{}

\cventry{2019}{Análisis de voladuras en minas subterráneas}{Inspección con dron Elios 2}{}{}{}

\cventry{2019}{Realizando trabajos de enlace con la geodesia espacial, poligonal y escaneado láser 3D}{Documentación geométrica de Polidux (Monzón)}{}{}{}

\cventry{2019}{Realizando trabajos de enlace con la geodesia espacial, poligonal y escaneado láser 3D}{Documentación geométrica de la Fuente del Berro (Madrid)}{}{}{}

\cventry{2019}{Realizando trabajos de enlace con la geodesia espacial, poligonal y escaneado láser 3D}{Documentación geométrica de Necrópolis Musulmana en }{Tudela, (Navarra)}{}{}

\cventry{2019}{Realizando trabajos de enlace con la geodesia espacial, poligonal y escaneado láser 3D}{Documentación geométrica de las instalaciones del FIDAMC (Madrid)}{}{}{}

\cventry{2018}{Realizando trabajos de nivelación geométrica, análisis de estructuras y asentamiento de edificios}{Auscultación del Colector de aguas fecales}{Santa Coloma de Barcelona}{}{}

\cventry{2018}{Reconstrucción tridimensional y fotorealística de escenarios de la serie “La Peste” para VR}{Fotogrametría y escaneado láser}{Sevilla}{}{}

\cventry{2018}{Reconstrucción tridimensional y fotorealística de una manyatta maasai para documental de realidad virtual}{Fotogrametría y escaneado láser}{}{}{}

\cventry{2018}{Realizando trabajos de enlace con la geodesia espacial, poligonal y escaneado láser 3D}{Documentación geométrica de Viajes de agua de Amaniel (Madrid)}{}{}{}

\cventry{2017}{Realizando trabajos de enlace con la geodesia espacial, poligonal y escaneado láser 3D}{Relevé 3D de certains secteurs de la grotte de Bruniquel (Francia)}{}{}{}

\cventry{2017}{Realizando trabajos de enlace con la geodesia espacial, poligonal y escaneado láser 3D}{Documentación geométrica de la Torre de Conchel (Huesca)}{}{}{}



\end{document}